% Unofficial University of Oxford Poster Template
% https://github.com/andiac/gemini-cam
% a fork of https://github.com/anishathalye/gemini

\documentclass[final]{beamer}

% ====================
% Packages
% ====================

\usepackage[T1]{fontenc}
\usepackage{lmodern}
\usepackage[size=custom,width=120,height=72,scale=1.0]{beamerposter}
\usetheme{gemini}
\usecolortheme{ox}
\usepackage{graphicx}
\usepackage{booktabs}
\usepackage{tikz}
\usepackage{pgfplots}
\pgfplotsset{compat=1.14}
\usepackage{anyfontsize}
\usepackage{colortbl} % pour surligner les lignes des tableaux
\usepackage{adjustbox}
\usepackage{minted} % pour ajouter du code
\usepackage{svg} % pour le QR code au format svg

% packages pour la bibliographie
\usepackage[style=numeric, maxnames=2, minnames=1, sorting = none]{biblatex} 
\addbibresource{poster.bib} 

% pour le diagramme
\usepackage{tikz}
\usetikzlibrary{shapes.geometric, arrows, fit}

% ====================
% Lengths
% ====================

% If you have N columns, choose \sepwidth and \colwidth such that
% (N+1)*\sepwidth + N*\colwidth = \paperwidth
\newlength{\sepwidth}
\newlength{\colwidth}
\setlength{\sepwidth}{0.025\paperwidth}
\setlength{\colwidth}{0.3\paperwidth}

\newcommand{\separatorcolumn}{\begin{column}{\sepwidth}\end{column}}



% ====================
% Title
% ====================

\title{Automatic analysis of metadata for secondary tabular data protection}

\author{Clara Baudry \inst{1}}

\institute[shortinst]{\inst{1} Insee (French National Statistical Institute)}

% ====================
% Footer (optional)
% ====================

\footercontent{
  \href{https://www.insee.fr}{https://www.insee.fr} \hfill
  NTTS Conference 2025, Brussels \hfill
  \href{mailto:clara.baudry@insee.fr}{clara.baudry@insee.fr}}
% (can be left out to remove footer)

% ====================
% Logo (optional)
% ====================
% use this to include logos on the left and/or right side of the header:
\logoleft{\includegraphics[height=7cm]{logos/Logo_Insee_SignatureAdroite.png}}
% \logoleft{\includegraphics[height=7cm]{logos/ox-logo.png}}

\logoright{\includesvg[height=7cm]{images/qr_code.svg}}

% ====================
% Body
% ====================

\begin{document}



\begin{frame}[fragile] % pour pouvoir mettre le code avec minted
\begin{columns}[t]
\separatorcolumn

\begin{column}{\colwidth}

  \begin{block}{Published tables $\neq$ protected tables}

    The R package \texttt{rtauargus} allows users to call Tau-Argus from R. It leverages the renowned SDC software Tau-Argus to automatically handle linked tables, extending its functionality beyond separate table processing \cite{VignetteLinkedTables}. It detects common cells between tables and flags it accordingly. 
    
    Nevertheless, for the package to detect common cells, a list of linked tables must be provided. The constitution of this list of tables demands a certain level of expertise in the understanding of how tables are linked from a confidentiality viewpoint \cite{HandbookSDC}. 
    
    Many statisticians in national institutes end up protecting tables only once a year, thus they hardly get to reach the level of expertise needed. The tool helps all staff go from the published tables metadata to the tables needing protection metadata.

  \end{block}

  \begin{block}{Eurostat template}

    The template is a \texttt{.csv} file giving the structure of the \texttt{XML} files sent by the NSIs to Eurostat. Some columns are pre-filled, describing all the cells expected by Eurostat. The analysis only focuses on these columns. The idea is to deduct from the cells description the tables to protect.
     \begin{table}
      \centering
      \begin{tabular}{r r r r r}
        \toprule
        \textbf{TIME\_PERIOD} & \textbf{INDICATOR} & \textbf{ACTIVITY} & \textbf{NUMBER\_EMPL} & \textbf{LEGAL\_FORM} \\
        \midrule
        2022 & SAL & B & E0 & \_T \\
        2022 & SAL & B & E1T4 & \_T \\
        ... & ... & ... & ... & ... \\
        2022 & SAL\_DTH & BTSXO\_S94 & E0 & \_T \\
        2021 & SAL\_DTH & BTSXO\_S94 & E1T4 & \_T \\
        ... & ... & ... & ... & ... \\
        \bottomrule
      \end{tabular}
      \caption{Extract from a SBS Eurostat template}
      \label{meta_pizza_lettuce}
    \end{table}

    \begin{minipage}{0.6\textwidth}  % First image on the left
    \begin{minted}
    [
    frame=lines,
    framesep=2mm,
    baselinestretch=1.2,
    linenos,
    bgcolor=insee_light_grey
    ]
    {R}
    template_formatted <- format_template(
      data = enterprise_template,
      indicator_column = "INDICATOR",
      spanning_var_tot = list(
        ACTIVITY = "BTSXO_S94",
        NUMBER_EMPL = "_T",
        LEGAL_FORM = "_T"),
      field_columns = c("TIME_PERIOD")
    )
    \end{minted}
    \end{minipage}%
    \hfill
    \begin{minipage}{0.35\textwidth}
        In \texttt{rtauargus} \cite{githubrepo} a function \texttt{template\_formatted()} is being developed. It automatically analyses the cells of the file and returns the list of tables to protect. \\
        The user needs to classify the variables: indicator, spanning and field variables. For the spanning variables, it is necessary to specify the modality corresponding to the total. 
    \end{minipage}

    \vspace{0.5cm}
    
    Table 2 shows the output of this function: a dataframe describing the tables that need protection (in the format of the input expected by the metadata analysis tool).

    \vspace{-0.5cm}

    \begin{table}
      \centering
      \begin{tabular}{l l l l l l l}
        \toprule
        \textbf{table\_name} & \textbf{field} & \textbf{indic} & \textbf{span\_1} & \textbf{span\_2} & \textbf{hrc\_span\_1} & \textbf{hrc\_span\_2} \\
        \midrule
        2021\_SAL\_DTH\_1 & 2021 & SAL\_DTH & ACT & LEGAL & hrc\_activity\_131 & hrc\_legal\_3 \\
        2021\_SAL\_DTH\_2 & 2021 & SAL\_DTH & ACT & NUMBER\_EMPL & hrc\_activity\_131 & hrc\_number\_4 \\
        2022\_SAL\_1 & 2022 & SAL & ACT & LEGAL & hrc\_activity\_131 & hrc\_legal\_3 \\
        2022\_SAL\_2 & 2022 & SAL & ACT & NUMBER\_EMPL & hrc\_activity\_131 &  hrc\_number\_4\\
        2022\_SAL\_DTH\_1 & 2022 & SAL\_DTH & ACT & LEGAL & hrc\_activity\_131 & hrc\_legal\_3 \\
        2022\_SAL\_DTH\_2 & 2022 & SAL\_DTH & ACT & NUMBER\_EMPL & hrc\_activity\_131 & hrc\_number\_4 \\
        \bottomrule
      \end{tabular}
      \caption{SBS metadata ready for automatic analysis}
      \label{meta_pizza_lettuce}
    \end{table}

    
  \end{block}

\end{column} % fin de la première colonne du poster

\separatorcolumn

\begin{column}{\colwidth}

  \begin{alertblock}{Automatic analysis of metadata}

    \begin{enumerate}
      \item \textbf{Identify hierarchies}. All the spanning variables included in the same hierarchical relationship are renamed after the latter. Thus, all those variables are grouped in one dimension. 
      \item \textbf{Split in clusters}. Gather tables in groups that should be protected together because they share common cells. Those groups are called clusters. Each cluster is independent from the others.
      \item \textbf{Detect tables included in other tables}. Analyse the spanning variables to detect tables included in other tables.
      \item \textbf{Regroup tables that are included in each other}. Regroup tables that are included in each other in one table to treat.
      \item \textbf{Provide dataframe with cluster assignment}. Each table needing protecting is described and assigned to a cluster. The tables assigned to the same clusters share common cells and must be treated jointly, using for example \texttt{rtauargus::tab\_multi\_manager()}.
    \end{enumerate}

  \end{alertblock}

  \begin{block}{Theoretical example: tables on pizza and lettuce turnover}

    The input is a dataframe containing the metadata. Table \ref{meta_pizza_lettuce} shows an example of the said dataframe. It describes the tables of the turnover from the sales of pizzas and lettuces that will be published.

    \begin{table}
      \centering
      \begin{tabular}{r r r r r r r r r}
        \toprule
        \textbf{table} & \textbf{field} & \textbf{hrc\_field} & \textbf{indic} & \textbf{hrc\_indic} & \textbf{span\_1} & \textbf{hrc\_span\_1} & \textbf{span\_2} & \textbf{hrc\_span\_2} \\
        \midrule
        \rowcolor{insee_light_red} T1 & 2023 & \textit{NA} & to\_pizza & \textit{NA} & nuts2 & hrc\_nuts & size & \textit{NA} \\
        \rowcolor{insee_light_red} T2 & 2023 & \textit{NA} & to\_pizza & \textit{NA} & nuts3 & hrc\_nuts & size & \textit{NA} \\
        \rowcolor{insee_light_red} T3 & 2023 & \textit{NA} & to\_pizza & \textit{NA} & act & hrc\_nace & nuts2 & hrc\_nuts \\
        \rowcolor{insee_light_red} T4 & 2023 & \textit{NA} & to\_pizza & \textit{NA} & act & hrc\_nace & nuts3 & hrc\_nuts \\
        \rowcolor{insee_light_green} T5 & 2023 & \textit{NA} & to\_batavia & hrc\_lettuce & act & hrc\_nace & size & \textit{NA} \\
        \rowcolor{insee_light_green} T6 & 2023 & \textit{NA} & to\_arugula & hrc\_lettuce & act & hrc\_nace & size & \textit{NA} \\
        \rowcolor{insee_light_green} T7 & 2023 & \textit{NA} & to\_lettuce & hrc\_lettuce & act & hrc\_nace & size & \textit{NA} \\
        \bottomrule
      \end{tabular}
      \caption{Pizza and lettuce turnover metadata}
      \label{meta_pizza_lettuce}
    \end{table}

    \begin{minipage}{0.75\textwidth}
    \begin{minted}
    [
    frame=lines,
    framesep=2mm,
    baselinestretch=1.2,
    linenos,
    bgcolor=insee_light_grey
    ]
    {R}
    res <- analyse_metadata(
            df_metadata = meta_pizza_lettuce,
            verbose = TRUE # to get all the steps
    )          
    \end{minted}
    \end{minipage}%
    \hfill
    \begin{minipage}{0.18\textwidth}
        \includegraphics[height = 6cm]{logos/rtauargus_logo.png}
    \end{minipage}

  \end{block}

  \begin{block}{Identify hierarchies and split in clusters}
      The tool renames all the variables by the name of their hierarchies. The confidentiality approach is in terms of hierarchies more than variables. Then, it analyses the fields and indicators in order to determine which tables are independent from each other. Tables are independent when they share no common cells. In this example, \textbf{2 clusters} are detected.
    
    \vspace{0.7cm}
    
    \begin{figure}[ht]
    \centering
    \begin{minipage}{0.48\textwidth}  % First image on the left
        \centering
        \includegraphics[height = 4.7cm]{images/pizza.png}
        \vspace{0.6cm}
        
        The tables T1, T2, T3 and T4 all refer to pizza turnover, they need to be treated together.
    \end{minipage}%
    \hfill
    \begin{minipage}{0.48\textwidth}  % Second image on the right
        \centering
        \includegraphics[height = 5cm]{images/salade.png}
        \vspace{0.3cm}
        
        The tables T5, T6 and T7 all refer to lettuce turnover, they need to be treated together.
    \end{minipage}
    \end{figure}

      
  \end{block}

\end{column} % fin de la deuxième colonne du poster

\separatorcolumn

\begin{column}{\colwidth}

  \begin{block}{Tables included in each other}
    The tool analyses the spanning variables to detect tables included in other tables.
  
    \begin{figure}[ht]
    \centering
    \begin{minipage}{0.48\textwidth}  % First image on the left
        \centering
        \includegraphics[height = 9.8cm]{images/graphes_to_pizza_rogne.png}
        \caption{Cluster \alpha : tables on pizza turnover}
        \label{graph_pizza}
        Only \textbf{2 tables} are needed in order to protect all the cells counting pizza turnover.
    \end{minipage}%
    \hfill
    \begin{minipage}{0.48\textwidth}  % Second image on the right
        \centering
        \includegraphics[height = 10cm]{images/graphe_to_lettuce_rogne.png}
        \caption{Cluster \beta : tables on lettuce turnover}
        \label{graph_lettuce}
        Only \textbf{1 table} is needed in order to protect all the cells counting lettuce turnover.
    \end{minipage}
    \end{figure}

  \end{block}

  \begin{block}{Result of the analysis}

    Once the tool has gone through all stages it returns a dataframe with a cluster assignment variable.

    \vspace{-0.5cm}
    
    \begin{table}
      \centering
      \begin{tabular}{r r r r r r r r r}
        \toprule
        \textbf{cluster} & \textbf{table} & \textbf{indic} & \textbf{span\_1} & \textbf{span\_2} & \textbf{span\_3} & \textbf{hrc\_span\_1} &  \textbf{hrc\_span\_2} & \textbf{hrc\_span\_3} \\
        \midrule
        \rowcolor{insee_light_green} $\beta$ & T5.T6.T7 & \textbf{lettuce} & \textbf{hrc\_nace} & size & \textbf{hrc\_lettuce} &  &  & hrc\_lettuce \\
        \rowcolor{insee_light_red} $\alpha$ & T1.T2 & to\_pizza & \textbf{hrc\_nuts} & size &  & hrc\_nuts &  &  \\
        \rowcolor{insee_light_red} $\alpha$ & T3.T4 & to\_pizza & \textbf{hrc\_nace} & \textbf{hrc\_nuts} &  & hrc\_nace  &  hrc\_nuts &   \\
        \bottomrule
      \end{tabular}
      \caption{Result of the analysis: metadata of the tables to protect}
    \end{table}

    The analysis shows that in order to protect the 7 tables to be published only 3 tables need to be protected. Moreover, 2 of them need a joint treatment which can be performed like so:

    \begin{minipage}{1\textwidth}
    \begin{minted}
    [
    frame=lines,
    framesep=2mm,
    baselinestretch=1.2,
    linenos,
    bgcolor=insee_light_grey
    ]
    {R}
    safe_tables <- tab_multi_manager(
      list_tables = list(T1T2 = pizza_nuts_size,
                         T3T4 = pizza_nace_nuts),
      list_explanatory_vars = list(T1T2 = c("NUTS", "size"),
                                   T3T4 = c("act", "NUTS")),
      hrc = c(NUTS = "hrc_nuts.hrc", act = "hrc_nace.hrc"),
      freq = "N_OBS", value = "to", totcode =  "Total",
      secret_var = "is_secret_prim")         
    \end{minted}
    \end{minipage}

  \end{block}

\vspace{-0.5cm}

  \begin{block}{References}
    \printbibliography
  \end{block}

\end{column}

\separatorcolumn
\end{columns}
\end{frame}

\end{document}
